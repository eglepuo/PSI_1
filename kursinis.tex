\documentclass{VUMIFPSkursinis}
\usepackage{algorithmicx}
\usepackage{algorithm}
\usepackage{algpseudocode}
\usepackage{amsfonts}
\usepackage{amsmath}
\usepackage{bm}
\usepackage{caption}
\usepackage{color}
\usepackage{float}
\usepackage{graphicx}
\usepackage{listings}
\usepackage{subfig}
\usepackage{wrapfig}
\usepackage{sectsty}

\usepackage{enumitem}
%PAKEISTA, tarpai tarp sąrašo elementų
\setitemize{noitemsep,topsep=0pt,parsep=0pt,partopsep=0pt}
\setenumerate{noitemsep,topsep=0pt,parsep=0pt,partopsep=0pt}
\allsectionsfont{\centering}
% Titulinio aprašas
\university{Vilniaus universitetas}
\faculty{Matematikos ir informatikos fakultetas}
\department{Programų sistemų katedra}
\papertype{Programų sistemų inžinerijos I laboratorinis darbas}
\title{Socialinis Vilniaus universiteto tinklalapis}
\titleineng{SocialVU}
\status{2 kurso 4 grupės studentai}
\author{Andrejus Voitovas}
\secondauthor{Eglė Puodžiūnaitė}
\thirdauthor{Kasparas Kralikas}
\fourthauthor{Ieva Vizgirdaitė} % Pridėti antrą autorių
\supervisor{asist. dr. Vytautas Valaitis}
\date{Vilnius – \the\year}

% Nustatymai
% \setmainfont{Palemonas}   % Pakeisti teksto šriftą į Palemonas (turi būti įdiegtas sistemoje)
\bibliography{bibliografija}

\begin{document}
% PAKEISTA
\maketitle
\cleardoublepage\pagenumbering{arabic}
\setcounter{page}{2}
\sectionnonum{ANOTACIJA}
Šiame dokumente pateikiami funkciniai ir nefunkciniai reikalavimai sistemai. Sistema ana-
lizuojama taikant ICONIX metodą. Apibrėžiamas struktūrinis dalykinės srities modelis, paaiški-
namos sistemoje naudojamos sąvokos. Taip pat aprašomos sistemoje atliekamos užduotys, anali-
zuojami pagrindiniai ir alternatyvūs užduoties scenarijai, naudojant sekų diagramas. Apibrėžiama
techninė kuriamos sistemos architektūra bei testavimo planas ir scenarijai.
\newpage
%TURINYS
\tableofcontents

\sectionnonum{ĮVADAS}
\textbf{Tikslas}  - sukurti socialinio tinklalalapio prototipą, kurį įgyvendinus būtų palengvinta universiteto bendruomenės komunikaciją.\\
\textbf{Temos aktualumas} \\
Šiuo metu studentams dėstytojų skelbiama informacija yra išbarstyta internete, kurią surasti užima galybes laiko. Yra atskiras universiteto naujienų puslapis, kiekvienas dėstytojas turi savo asmeninį tinklalapį, atskiras elektroninis paštas. Tiek dėstytojui pasiekti studentus, tiek studentui dėstytoją yra komplikuota ir nepatogu.\\
\textbf{Dalykinė sritis}\\
Socialinis Vilniaus Universiteto tinklapis.\\
\textbf{Probleminė sritis}\\
Socialinis Vilniaus Universiteto tinklapis suteiktų galimybę greitai ir paprastai pasiekti šio universiteto dėstytojų puslapius, informaciją juose, susisiekti sus pačiais dėstytojais. Pagrindinis tinklalapio išskirtinumas - greitai ir patogiai pasiekiama informacija, viskas vienoje vietoje. Itin patogus valdymas dėstytojams.\\
 \textbf{Naudoti dokumentai}\\
 Dokumentas parengtas pagal kursinio darbo reikalavimus naudojant Latex programą ir jau sukurtus šablonus.\\
 \textbf{Darbo pagrindas} \\ 
Dokumentas parengtas kaip Programų sistemų inžinerijos I laboratorinis darbas.
\newpage
\sectionnonum{KURIAMOS SISTEMOS ARCHITEKTŪRA}
EGLĖ
\newpage
\section{LOGINIS PJŪVIS}
Loginį pjūvį sudaro klasių diagramos, kurios naudojamos pavaizduoti sistemos architektūros projektavimo etapus.
\subsection{Esybių klasių diagrama (nulinis lygis)}
\begin{figure}[H]
\centering
\includegraphics[width=\linewidth]{img/dalykine.png}
\label{fig:dalykine}
\caption{Dalykinė srities UML diagrama}
\end{figure}
Esybių diagramoje \ref{fig:dalykine} vaizduojamos esybių sąsajos. Pagrindinė esybė Naudotojas,
kuris gali būti Studentas, Dėstytojas arba Administratorius. Studentas turi galimybę naudotis pagrindinėmis puslapio funkcijos, o dėstytojai ir administratoriai pateikti naudingą studentams meždiagą. Taip pat studentai bei dėstytojai gali komunikuoti tarpusavyje nesinaudojant trečiųjų šalių komunikacinėmis priemonėmis. Administratoriai, savo ruožtu, pateikia informaciją apie renginius, naujienas ir D.U.K.
\subsection{Klasių diagrama (pirmas lygis)}
\begin{figure}[H]
\centering
\includegraphics[width=\linewidth]{img/pagrindine.png}
\label{fig:pagrindine}
\caption{Dalykinė srities UML diagrama}
\end{figure}
Pagrindinį programos funkcionalumą užtikrina šios klasės: Studentas, Dėstytojas, Administratorius, Reitingas, Dėsytotojo puslapis, Naujienos, Autentifikacija, Duomenų bazė,
D.U.K., Renginiai. Veikimą įgyvendinačių klasių tarpusavio bendradarbiavimas vaizduojamas asociacija, ge-
neralizacija, kompozicija bei kardinalumus \ref{fig:pagrindine}.
\newpage
\section{UŽDUOČIŲ PJŪVIS}
IEVA
\newpage
\section{KŪRIMO PJŪVIS}
KASPARAS
\newpage
\section{FIZINIS PJŪVIS}
KASPARAS
\newpage
\section{PROCESO PJŪVIS}
EGLĖ
\newpage
\section{RYŠIAI TARP PJŪVIŲ}
EGLĖ
\newpage
\sectionnonum{IŠVADOS}
KASNORS
\newpage
\sectionnonum{ŠALTINIAI}
KASNORS
\end{document}