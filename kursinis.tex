\documentclass{VUMIFPSkursinis}
\usepackage{algorithmicx}
\usepackage{algorithm}
\usepackage{algpseudocode}
\usepackage{amsfonts}
\usepackage{amsmath}
\usepackage{bm}
\usepackage{caption}
\usepackage{color}
\usepackage{float}
\usepackage{graphicx}
\usepackage{listings}
\usepackage{subfig}
\usepackage{wrapfig}
\usepackage{sectsty}

\usepackage{enumitem}
%PAKEISTA, tarpai tarp sąrašo elementų
\setitemize{noitemsep,topsep=0pt,parsep=0pt,partopsep=0pt}
\setenumerate{noitemsep,topsep=0pt,parsep=0pt,partopsep=0pt}
\allsectionsfont{\centering}
% Titulinio aprašas
\university{Vilniaus universitetas}
\faculty{Matematikos ir informatikos fakultetas}
\department{Programų sistemų katedra}
\papertype{Programų sistemų inžinerijos I laboratorinis darbas}
\title{Socialinis Vilniaus universiteto tinklalapis}
\titleineng{SocialVU}
\status{2 kurso 4 grupės studentai}
\author{Andrejus Voitovas}
\secondauthor{Eglė Puodžiūnaitė}
\thirdauthor{Kasparas Kralikas}
\fourthauthor{Ieva Vizgirdaitė} % Pridėti antrą autorių
\supervisor{asist. dr. Vytautas Valaitis}
\date{Vilnius – \the\year}

% Nustatymai
% \setmainfont{Palemonas}   % Pakeisti teksto šriftą į Palemonas (turi būti įdiegtas sistemoje)
\bibliography{bibliografija}

\begin{document}
% PAKEISTA
\maketitle
\cleardoublepage\pagenumbering{arabic}
\setcounter{page}{2}
\sectionnonum{ANOTACIJA}
ANDRIUS
\newpage
%TURINYS
\tableofcontents

\sectionnonum{ĮVADAS}
\textbf{Tikslas}  - sukurti socialinio tinklalalapio prototipą, kurį įgyvendinus būtų palengvinta universiteto bendruomenės komunikaciją.\\
\textbf{Temos aktualumas} \\
Šiuo metu studentams dėstytojų skelbiama informacija yra išbarstyta internete, kurią surasti užima galybes laiko. Yra atskiras universiteto naujienų puslapis, kiekvienas dėstytojas turi savo asmeninį tinklalapį, atskiras elektroninis paštas. Tiek dėstytojui pasiekti studentus, tiek studentui dėstytoją yra komplikuota ir nepatogu.\\
\textbf{Dalykinė sritis}\\
Socialinis Vilniaus Universiteto tinklapis.\\
\textbf{Probleminė sritis}\\
Socialinis Vilniaus Universiteto tinklapis suteiktų galimybę greitai ir paprastai pasiekti šio universiteto dėstytojų puslapius, informaciją juose, susisiekti sus pačiais dėstytojais. Pagrindinis tinklalapio išskirtinumas - greitai ir patogiai pasiekiama informacija, viskas vienoje vietoje. Itin patogus valdymas dėstytojams.\\
 \textbf{Naudoti dokumentai}\\
 Dokumentas parengtas pagal kursinio darbo reikalavimus naudojant Latex programą ir jau sukurtus šablonus.\\
 \textbf{Darbo pagrindas} \\ 
Dokumentas parengtas kaip Programų sistemų inžinerijos I laboratorinis darbas.
\newpage
\sectionnonum{KURIAMOS SISTEMOS ARCHITEKTŪRA}
EGLĖ
\newpage
\section{LOGINIS PJŪVIS}
ANDRIUS
\newpage
\section{UŽDUOČIŲ PJŪVIS}
IEVA
\newpage
\section{KŪRIMO PJŪVIS}
KASPARAS
\newpage
\section{FIZINIS PJŪVIS}
KASPARAS
\newpage
\section{PROCESO PJŪVIS}
EGLĖ
\newpage
\section{RYŠIAI TARP PJŪVIŲ}
EGLĖ
\newpage
\sectionnonum{IŠVADOS}
KASNORS
\newpage
\sectionnonum{ŠALTINIAI}
KASNORS
\end{document}